\input{_includes/header}

% MetaData
\newcommand{\IUMSauthor}{Vorname Nachname}
\newcommand{\IUMSmatNr}{012345678}

\newcommand{\IUMStype}{ART DES LEISTUNGSNACHWEISES\\ -BITTE ART UND BEZEICHNUNG ENTSPRECHEND DEN ANGABEN LEISTUNGSNACHWEISE IM EIS VERWENDEN} % Master-Thesis/ Bachelor-Thesis
\newcommand{\IUMStitle}{DER LANGE TITEL DER\\ WISSENSCHAFTLICHEN ARBEIT}
\newcommand{\IUMSsubtitle}{IM RAHMEN DES PROJEKT-KOMPETENZ-STUDIUMS\\ \vspace{0.25em} ZUM }
\newcommand{\IUMSsubmissionMonth}{21. Dezember 2017} % z.B. \today
\newcommand{\IUMSsubmissionDate}{21. Dezember 2017} % z.B. \today
\newcommand{\SCMTstartevent}{30. Oktober 2020} % z.B. \today
\newcommand{\SCMTendevent}{30. Oktober 2020} % z.B. \today
\newcommand{\SCMTdash}{ - }
\newcommand{\SCMTlocation}{Stuttgart}
\newcommand{\SCMTstudyprogramlarge}{MASTER OF WIRTSCHAFTSINFORMATIK }
\newcommand{\SCMTstudyprogramregular}{Master of Wirtschaftsinformatik }
\newcommand{\SCMTsupervisor}{Akad. Grad Vorname Nachname Hochschule}
\newcommand{\companyname}{<Firma> }
\newcommand{\companysupervisor}{Akad. Grad Vorname Nachname Unternehmen}
\newcommand{\companyprojectleader}{<Titel>/<Name>/<Position>/<Abteilung>}
\newcommand{\SCMTdatetime}{Ort, Datum}
\newcommand{\SCMTsignature}{Unterschrift}
\newcommand{\SCMTsecretdocument}{Vertrauliches Dokument}
% End MetaData

% Uncomment for print version
%\printoutput

% Bib-File
%\bibliographystyle{stylename}
\bibliography{bibliography/literatur}

\begin{document}

\hypersetup{pageanchor=false}
\begin{titlepage}
\begin{figure}[htbp]
  \begin{minipage}[t][2cm][c]{\textwidth/2+1.7cm}
	
		\includegraphics[height=4em]{_includes/images/scmt-logo.png}  

  \end{minipage}
  \begin{minipage}[t][3cm][c]{\textwidth/2-1.7cm}
	\begin{flushright}
  %	\includegraphics[height=5em]{_includes/images/<>.jpeg}
	\end{flushright}
  \end{minipage}
\end{figure}



\vspace{3em}
\begin{center}
	\large{\textsf{\SCMTtype}}
\end{center}

\vspace{2em}
\begin{figure}[htbp]
\centering
\includegraphics[height=9em]{_includes/images/some-company-logo.png}
\end{figure}

\begin{center}
\vspace{2em}
\Huge{\textbf{\textsf{\SCMTtitle}}}


\large{\emph{\textsf{\SCMTsubtitle\SCMTstudyprogramlarge}}}
\vspace{\fill}
\end{center}

\vspace{1.5em}
\normalsize 
\begin{center}
	\textbf{\textsf{Autor:}}\\ \vspace{0.25em}
	\SCMTauthor\\ \vspace{0.25em}
	Durchführung, Mat.-Nr.: \SCMTmatNr %Durchführung ist Semester Jahrgang "Winf 05"	
\end{center}
%\begin{flushleft}
%
%\begin{figure}[b]
%\centering
%\begin{minipage}[b]{\textwidth/2-0.1cm+0.85cm}
%\begin{flushleft}
%\SCMTsubmissionMonth\\ \vspace{1em}
%\SCMTauthor\\ \vspace{0.25em}
%Mat.-Nr.: \SCMTmatNr
%\end{flushleft}
%\end{minipage}
%\begin{minipage}[b]{\textwidth/2-0.1cm-0.85cm}
%\begin{flushleft}
%\textbf{\textsf{Betreuer}}\\ \vspace{0.25em}
%\SCMTsupervisor\\
%\vspace{1em}
%\textbf{\textsf{Verantwortl. Hochschullehrer}}\\ \vspace{0.25em}
%Akad. Grad Vorname Nachname
%\end{flushleft}
%\end{minipage}
%\end{figure}

%\end{flushleft}

\end{titlepage}

%\thispagestyle{empty}
%\cleardoublepage

\hypersetup{pageanchor=true}
\pagenumbering{roman}

\include{_includes/subtitlepage}

\include{_includes/erklaerung}

\include{_includes/sperrvermerk}

\include{_includes/permission}

\linespread{1.13}

% \pdfbookmark[1]{Inhaltsverzeichnis}{toc}
\tableofcontents					% create a toc
%\addcontentsline{toc}{chapter}{Inhaltsverzeichnis}
% wenn eine Seite beim Inhaltsverzeichnis fehlt (also zB drei statt vier Seiten lang ist):
% \thispagestyle{empty}
%\cleardoublepage

\linespread{1}
%\pagenumbering{arabic}
%
%\include{acronyms/acronymsTest}
% bei vierseitigen Inhaltsverzeichnis erst im Kapitel 1
%\pagenumbering{arabic}

%%%%% ABBILDUNGSVERZEICHNIS %%%%%
\listoffigures
%\addcontentsline{toc}{chapter}{Abbildungsverzeichnis}

%%%%% TABELLENVERZEICHNIS %%%%%
\listoftables

%%%%%  GLOSSARY  %%%%%%
%\printglossaries[type=acronym]
\printglossary[title={Abkürzungsverzeichnis}] %Generate List of Abbreviations

%%%%% QUELLCODEVERZEICHNIS %%%%%
\renewcommand\lstlistlistingname{Programmcodeverzeichnis}
\lstlistoflistings

\newpage
\cleardoublepage

\linespread{1}
\pagenumbering{arabic}
%%%%% INHALTSKAPITEL %%%%%
% bei einem vierseitigen Inhaltsverzeichnis darf die Nummerierung mit 1,2, .... erst hier beginnen, in den folgenden Kapiteln entfällt der Befehl
% bei einem vierseitigen Inhaltsverzeichnis diesen Befehl nicht auskommentieren
%\pagenumbering{arabic}

\chapter{Einleitung}\label{chap:einleitung}

Der grobe Aufbau und die Gliederung dieses Dokumentes entspricht dem typischen Aufbau einer studentischen Arbeit.

\section{Motivation}\label{sec:motivation}
Hier soll stehen, warum die in der Arbeit behandelten Konzepte, Betrachtungen und Lösungen gebraucht werden und praktisch relevant sind.

\section{Problemstellung und Zielsetzung}\label{sec:zielsetzung}
Was soll mit der Arbeit erreicht werden?

\section{Aufbau der Arbeit}\label{sec:aufbau}
Hier steht, was in den restlichen Kapiteln behandelt wird. Die Einleitung endet mit diesem Abschnitt.

%\pagenumbering{arabic}
\include{chapters/2_sota}
\include{chapters/3_anforderungen}
\chapter{Konzeption}\label{chap:konzeption}

In diesem Kapitel erfolgt die Darstellung des neuen, eigenen Konzeptes, welches es in dieser Form bisher noch nicht gibt. Dabei muss nachvollziehbar sein, wie die erarbeitete eigene Leistung sich in das Thema und das Forschungsgebiet einordnet. Alles, was in der Arbeit beschrieben wird, muss einen Bezug zum Thema bzw. zum vorgestellten Konzept haben. Werden für das Konzept Teile anderer Lösungen bzw. Ansätze verwendet oder weiterentwickelt, so ist dies deutlich von der eigenen Leistung abzugrenzen.

\chapter{Implementation}\label{chap:implementation}

Hier soll eine kurze Beschreibung der Herausforderungen und der wichtigen Eckdaten der Implementation des zuvor präsentierten Konzeptes vorgenommen werden. Gegebenenfalls wird hier noch auf die Evaluation der Implementation eingegangen und deren Messwerte interpretiert.
\begin{lstlisting}[language=java, numbers=none, caption=Exemplarisches Beobachter-Muster]
public enableProcessDisruptionSubject: Subject<THaDisruption> = new Subject();  
public enableProcessDisruptionObservable: Observable<THaDisruption> = 		  this.enableProcessDisruptionSubject.asObservable(); 
\end{lstlisting}

\include{chapters/6_evaluation}
\chapter{Zusammenfassung und Ausblick}\label{chap:zusaus}

Nochmal alles, was wichtig war wird hier erwähnt und der Bezug zur Zielstellung und Motivation wird hergestellt. Es sollte auf wichtige Fragestellungen, die nicht betrachtet wurden, aufmerksam gemacht und mögliche Ansätze bzw. Strategien für weiterführende Arbeiten aufgezeigt werden.

%zusätzliche leere Seite anfügen, da die Zusammenfassung nur eine Seite lang ist, aber der Anhang auf einer ungeraden Seite beginnen soll
%\thispagestyle{empty}
%\cleardoublepage

\thispagestyle{empty}
%\nocite{*}

%%%%% BIBLIOGRAFIE %%%%%%
%\shorthandoff{"}
\printbibliography[title={Literaturverzeichnis}]
%\shorthandon{"}

%%%%%%%%%%%%%%%%%%%%%%%%%%% ANHANG %%%%%%%%%%%%%%%%%%%%%%%%%%%%%%%%%%%
%\begin{appendix}
\appendix
%\pagenumbering{roman}
\input{chapters/a_anhang}



%\end{appendix}

\end{document}