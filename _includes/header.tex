\documentclass[DIV=10, parskip=full*, headings=big, twopage, pdftex, 11pt, cleardoublepage=plain,
bibliography=totoc, listof=totoc, BCOR=12mm,
numbers=noenddot]{scrbook}

\usepackage[ngerman]{babel}

\usepackage{helvet}
%\usepackage[a4paper, left=4cm, right=2cm, top=2cm, bottom=2cm]{geometry}
\usepackage[utf8]{inputenc}
\usepackage[T1]{fontenc} % for correct guillemets, http://en.wikibooks.org/wiki/LaTeX/Internationalization#German
\usepackage{lmodern} % to avoid rastered fonts, http://www.komascript.de/node/699
\usepackage{scrhack}

\usepackage[babel,german=quotes]{csquotes}
\usepackage[style=authoryear,backref=true,minnames=3,abbreviate=false]{biblatex}

%Paket laden
\usepackage[automake, acronym, toc]{glossaries}
%\usepackage[
%nonumberlist, %keine Seitenzahlen anzeigen
%acronym,      %ein Abkürzungsverzeichnis erstellen
%toc,          %Einträge im Inhaltsverzeichnis
%section]      %im Inhaltsverzeichnis auf section-Ebene erscheinen
%{glossaries}
%\makeglossaries
\makeglossaries
\newglossaryentry{RCP}{name={RCP},description={Rich Client Platform}}
\newglossaryentry{XPath}{name={XPath},description={XML Path Language}}
\newglossaryentry{XHTML}{name={HTML},description={Extensible Hypertext Markup Language}}
\newglossaryentry{SVN}{name={SVN},description={Subversion}}

\usepackage{booktabs}

\usepackage{typearea}

%Abk�rzungsverzeichnis
\usepackage{nomencl}
\renewcommand{\nomname}{Abkürzungsverzeichnis}
\setlength{\nomlabelwidth}{.20\hsize}
\renewcommand{\nomlabel}[1]{#1 \dotfill}

\makenomenclature

%Schriften
\usepackage[sc]{mathpazo}
\linespread{1.05}
\usepackage{microtype}
\usepackage{amssymb}

\usepackage{changebar}

% Palatino als Brotschrift
\fontfamily{phv}\fontseries{m}\selectfont

% AvantGarde f�r �berschriften
%\renewcommand{\familydefault}{\sfdefault} % Für die serifenlose Schriftart
\renewcommand{\sfdefault}{phv} % default f�r alles "serifenlose "--> AvantGarde
\setkomafont{subtitle}{\fontfamily{phv}\fontseries{m}\selectfont}
\setkomafont{subject}{\fontfamily{phv}\fontseries{m}\selectfont}
\setkomafont{pagehead}{\sffamily\footnotesize}
\setkomafont{pagenumber}{\sffamily\footnotesize}

\usepackage[onehalfspacing]{setspace}
%%%%%%%%%%%%% FONT TYPES %%%%%%%%%%%%%%
%ptm	Times
%phv	Helvetica
%pcr	Courier
%pbk	Bookman
%pag	Avant Garde
%ppl	Palatino
%bch	Charter
%pnc	New Century Schoolbook
%put	Utopia

% bold letters in equation
\usepackage{amsfonts}

%Reference https://www.physicsforums.com/threads/mathbb-stuff-isnt-working.544078/


\usepackage{xcolor}
\definecolor{darkgray}{gray}{0.2} 
\color{darkgray}
\definecolor{lightgrey}{rgb}{0.9, 0.9, 0.9} 
\definecolor{txtgreen}{rgb}{0.549, 0.749, 0.149}

\usepackage{amssymb, amsmath}
\usepackage[final]{pdfpages}

\usepackage{subfigure}
\usepackage{graphicx}
\usepackage{wrapfig}

\usepackage{flafter}
\usepackage{multicol}
\usepackage{multirow}
\usepackage{tabulary}
\usepackage{listings} % zum Einbringen von Code mit Syntaxhightlighting
\usepackage{blindtext}
\usepackage[perpage, ragged]{footmisc}

\definecolor{darkred}{rgb}{0.85, 0.3, 0.4}
\usepackage{courier}
\lstset{
	basicstyle=\small\ttfamily,
	keywordstyle=\color{blue},
	stringstyle=\color{darkred},
	commentstyle=\tiny\color{gray}
}
% voreingestellte Programmiersprache: Java
\lstset{
	language=Java, 
	breaklines=true, 
	breakautoindent=true, 
	breakatwhitespace=true, 
	numbers=left, 
	numberstyle=\tiny,
	extendedchars=true,
	rulesepcolor=\color{red},
	frame=lines,
	keywordstyle=\color{blue},
	commentstyle=\color{gray},
	stringstyle=\color{darkred},
	emph={double,bool,int,unsigned,char,true,false,void},
	emphstyle=\color{blue},
	emph={Assert,Test},
	emphstyle=\color{red},
	emph={[2]\using,\#define,\#ifdef,\#endif}, emphstyle={[2]\color{blue}},
	basicstyle=\small\ttfamily,
	showstringspaces=false,
	commentstyle=\tiny}


\usepackage[
pdftex,
hyperfootnotes=false,
bookmarks,
pdfpagelabels=true,
plainpages=false,
breaklinks]{hyperref} % f�r Verlinkungen in der PDF (extern und intern)


%colorlinks=false needs to be true in order to apply color
%\definecolor{lc}{cmyk}{0.6,0.16,1,0.015} % for screen output
%\definecolor{lc}{cmyk}{0.05,1.0,0.8, 0.0} % -red for screen output
%\definecolor{lc}{cmyk}{0.05,1.0,0.8, 0.2} % -red-dark1 for screen output
%\definecolor{lc}{cmyk}{0.05,1.0,0.8, 0.4} % -red-dark2 for screen output
%\definecolor{lc}{cmyk}{0.05,1.0,0.8, 0.6} % -red-dark3 for screen output
%\definecolor{lc}{cmyk}{1.0,0.0,0.55,0.2} % -green for screen output
%\definecolor{lc}{cmyk}{1.0,0.0,0.0,0.0} % -blue for screen output
%\definecolor{lc}{cmyk}{1.0,0.3,0.8,0.3} % -yellow for screen output
% turn on black link-color and set colorlink on false
%\definecolor{lc}{cmyk}{75,68,67,90} % -black for screen output
\newcommand{\printoutput}{\definecolor{lc}{cmyk}{0,0,0,1}} % for print output

%turn off link coloring by comment outlinkcolor downwards
\hypersetup{
	pdftoolbar=true,
	bookmarksopen,
	bookmarksnumbered=false,
	bookmarksopenlevel=1,
	pdfdisplaydoctitle,	
	colorlinks=false,
	linkcolor=lc,
	citecolor=lc,
	filecolor=lc,
	menucolor=lc,
	urlcolor=lc	
}
%\graphicspath{{images/}}

\usepackage[automark, footsepline]{scrlayer-scrpage}

%%%%%%%%%% Tabellenanpassungen %%%%%%%%%%%%%%%%%%
\setlength{\tabcolsep}{3pt}
\renewcommand{\arraystretch}{2}

\usepackage{tabularx}
\usepackage{array}
\usepackage{colortbl}
\usepackage{longtable}
\usepackage{caption}
\captionsetup{font=small}
\captionsetup{labelfont=bf}
\renewcommand{\tabularxcolumn}[1]{m{#1}} % Zellen von Tabellen immer vertikal zentrieren

%%%%%%%%%%%%%% Listen %%%%%%%%%%%%%%%%%%%%%
\usepackage{mdwlist}
\usepackage{setspace}
\renewcommand{\labelitemi}{\textperiodcentered}%{\sqbullet} %{\textperiodcentered}
\usepackage{enumitem}
\setlist{noitemsep}
\renewcommand{\labelenumi}{(\arabic{enumi})}

%%%%% Aufz�hlungsstyle %%%%%%%%
\newcommand{\myitem}[1]{\textsf{\textbf{#1}}.\hspace{1em}}

%%%%%%%%%% Pagina und Kolumnentitel %%%%%%%%%%%%%
\pagestyle{scrheadings}
\ohead{\headmark}
\ihead{\SCMTauthor}

\ofoot{\pagemark}
\ifoot{\SCMTsecretdocument}

\KOMAoptions{headsepline=0.5pt}
\KOMAoptions{footsepline=0pt}

\setcounter{secnumdepth}{3} % Tiefe der �berschriften, die Nummern bekommen
\setcounter{tocdepth}{2}    % Inhaltsverzeichnis bis Tiefe 3

%%%%%%%%%%%%%%%%%%%% Warnings %%%%%%%%%%%%%%%%%%%%%%%%
\pdfsuppresswarningpagegroup=1

%%%%%%%%%%%%%%%%%%%% TODO Notes %%%%%%%%%%%%%%%%%%%%%%%%
\usepackage[colorinlistoftodos]{todonotes} % TODO notes

\makeglossaries